\section{Method}
\label{sec:method}
\subsection{Pose Trust from an Inverse-Problem Perspective}
Conventional pose estimation predicts 2D joint coordinates directly from image features. 
However, prediction confidence alone does not reflect whether the estimated pose is supported by sufficient geometric evidence.

We instead ground trustworthiness in the geometry of 3D pose interpretation.
Any 2D pose implicitly asserts the existence of a physically plausible 3D configuration that explains the observation.
Trust therefore should not be measured solely by prediction accuracy, 
but by analyzing the properties of the inverse problem that recovers such a 3D explanation.

Let $\boldsymbol{\theta}$ denote the parameters of a kinematic human model.
A differentiable forward kinematics function produces 3D joint locations
\begin{equation}
\mathbf{P}(\boldsymbol{\theta}) = \mathrm{FK}(\boldsymbol{\theta}).
\end{equation}

Under a projection model $\Pi(\cdot)$, 2D observations satisfy
\begin{equation}
\mathbf{p} = \Pi(\mathbf{P}(\boldsymbol{\theta})).
\end{equation}

Given predicted 2D joints $\hat{\mathbf{p}}$, 
we recover a 3D explanation by solving
\begin{equation}
\boldsymbol{\theta}^{*}(\hat{\mathbf{p}})
=
\arg\min_{\boldsymbol{\theta}}
\left\|
\Pi(\mathbf{P}(\boldsymbol{\theta})) - \hat{\mathbf{p}}
\right\|_2^2
+
\lambda \|\boldsymbol{\theta}\|_2^2.
\label{eq:ik}
\end{equation}

This formulation exposes two fundamental properties of trust:
the existence of a plausible explanation and the stability of that explanation under perturbations.
\subsection{Solvability of the 3D Explanation}
The first aspect of trust concerns solvability.
A 2D pose prediction is unreliable if it cannot be explained by a physically plausible 3D configuration.

We evaluate the optimal reconstruction energy
\begin{equation}
E^{*}(\hat{\mathbf{p}})
=
\left\|
\Pi(\mathbf{P}^{*}) - \hat{\mathbf{p}}
\right\|_2^2
+
\lambda \|\boldsymbol{\theta}^{*}\|_2^2,
\end{equation}
where $\boldsymbol{\theta}^{*}$ solves Eq.~(\ref{eq:ik})
and $\mathbf{P}^{*} = \mathbf{P}(\boldsymbol{\theta}^{*})$.

Large reconstruction energy indicates geometric inconsistency between the predicted 2D joints and the kinematic model.
At the joint level, we decompose the reprojection residual:
\begin{equation}
e_j
=
\left\|
\Pi(\mathbf{P}^{*})_j - \hat{\mathbf{p}}_j
\right\|_2.
\end{equation}

We convert this residual into a solvability-based trust measure
\begin{equation}
\tilde{t}^{\mathrm{sol}}_j = \exp(-\alpha e_j).
\end{equation}

Solvability evaluates whether a plausible explanation exists, 
but it does not guarantee that the explanation is uniquely determined.


\subsection{Identifiability and Conditioning of the Inverse Problem}
Solvability evaluates whether a plausible 3D explanation exists.
However, existence alone does not guarantee that the explanation is uniquely determined by the observation.
In inverse problems, this distinction corresponds to the conditioning of the solution.

We analyze the local behavior of the optimization problem in Eq.~(\ref{eq:ik}).
Let
\begin{equation}
E(\boldsymbol{\theta}; \hat{\mathbf{p}})
=
\left\|
\Pi(\mathbf{P}(\boldsymbol{\theta})) - \hat{\mathbf{p}}
\right\|_2^2
+
\lambda \|\boldsymbol{\theta}\|_2^2.
\end{equation}
Assume that $\boldsymbol{\theta}^{*}$ is a local minimizer.
Under mild regularity conditions, the implicit function theorem implies that
$\boldsymbol{\theta}^{*}$ is locally a differentiable function of $\hat{\mathbf{p}}$,
and its sensitivity is governed by the Hessian of the objective.

\paragraph{Proposition 1 (Local Conditioning of the Inverse Problem).}
Let $\mathbf{J} = \frac{\partial \Pi(\mathbf{P}(\boldsymbol{\theta}))}{\partial \boldsymbol{\theta}}$
denote the Jacobian of the reprojection function evaluated at $\boldsymbol{\theta}^{*}$,
and let $\mathbf{H}$ denote the Hessian of $E$ with respect to $\boldsymbol{\theta}$.
If $\mathbf{H}$ is positive definite,
the sensitivity of the optimal solution satisfies
\begin{equation}
\frac{\partial \boldsymbol{\theta}^{*}}{\partial \hat{\mathbf{p}}}
=
\mathbf{H}^{-1} \mathbf{J}^\top.
\label{eq:implicit}
\end{equation}
Equation~(\ref{eq:implicit}) shows that the stability of the recovered 3D parameters
is governed by the inverse Hessian $\mathbf{H}^{-1}$.
When the Hessian is ill-conditioned or nearly singular,
small perturbations in the observed 2D joints can cause large variations in the optimal 3D solution.
Such configurations correspond to geometrically ambiguous or weakly constrained joints.

This observation provides a principled interpretation of trust:
a joint is unreliable when the inverse problem is locally ill-conditioned,
even if the reconstruction error is small.
Directly computing $\mathbf{H}^{-1}$ is computationally expensive.
Instead, we approximate conditioning at the joint level
by measuring the finite-difference sensitivity of the reconstructed 3D pose:
\begin{equation}
\mathbf{S}_j
\approx
\frac{
\left\|
\mathbf{P}^{*}(\hat{\mathbf{p}} + \epsilon \mathbf{e}_j)
-
\mathbf{P}^{*}(\hat{\mathbf{p}} - \epsilon \mathbf{e}_j)
\right\|_2
}{2\epsilon}.
\end{equation}
\subsection{Unified Trust Estimation}
Solvability and identifiability represent complementary aspects of trust.
Solvability evaluates geometric consistency,
while identifiability evaluates conditioning.\cite{}

We interpret trustworthy estimation as requiring both properties.
A joint is reliable only if it admits a plausible explanation 
and that explanation is stably determined by the observation.

We therefore combine the two signals conservatively:
\begin{equation}
\tilde{t}_j
=
\min
\left(
\tilde{t}^{\mathrm{sol}}_j,
\tilde{t}^{\mathrm{id}}_j
\right).
\end{equation}

A lightweight verifier head is trained to regress $\tilde{t}_j$
from intermediate pose features,
while the standard pose estimation loss is preserved.