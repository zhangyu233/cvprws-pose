\section{Related Work}
\label{sec:related}

\paragraph{Uncertainty Estimation in Pose Estimation.}
%
Modern 2D pose estimators,
from convolutional architectures such as
Stacked Hourglass~\cite{}, SimpleBaseline~\cite{}, and HRNet~\cite{}
to transformer-based designs including ViTPose~\cite{} and TokenPose~\cite{},
achieve strong accuracy on standard benchmarks
yet offer limited mechanisms for assessing prediction reliability.
The most common proxy is the peak response of predicted heatmaps~\cite{},
which reflects model confidence but not geometric validity.
At the aleatoric level,
sigma-regression heads~\cite{} learn per-joint variance alongside coordinates.
At the epistemic level,
Monte Carlo dropout~\cite{} and deep ensembles~\cite{}
estimate predictive dispersion through multiple stochastic or independent forward passes.
More recently, normalizing flows~\cite{} and diffusion-based models~\cite{}
have been proposed to capture multimodal joint distributions.
While these methods quantify dispersion within the model's representational space,
high confidence does not necessarily imply geometric validity,
and low variance does not guarantee physical consistency.
In particular, predictive uncertainty does not explicitly account for
whether a pose admits a stable 3D interpretation---the
question our framework is designed to answer.

\paragraph{Geometric and 3D-Constrained Pose Estimation.}
%
A large body of work incorporates 3D reasoning into pose estimation.
Direct 3D regression~\cite{} and graph-based lifting networks~\cite{}
recover 3D poses from 2D detections,
while parametric body models such as SMPL~\cite{} and STAR~\cite{}
provide a differentiable kinematic representation of the human body.
Inverse kinematics solvers,
including the differentiable formulation in HybrIK~\cite{},
and reprojection-based self-supervision~\cite{}
enforce geometric plausibility during training or inference.
Test-time augmentation~\cite{} and cycle-consistency checks~\cite{}
similarly exploit equivariance or reconstruction agreement
to refine or filter predictions.
These approaches improve accuracy
by constraining predictions to lie on a kinematic manifold.
However, their primary objective is refinement rather than reliability assessment.
They evaluate whether a pose is plausible,
but do not analyze the conditioning of the inverse mapping
from 2D observations to 3D articulation.
As a result, they cannot distinguish between well-identified solutions
and geometrically ambiguous configurations that happen to achieve low reprojection error.

\paragraph{Failure Detection and Risk-Aware Vision.}
%
Selective prediction,
originating from the ``reject option'' in classification~\cite{},
allows a model to abstain when uncertain.
Risk-coverage analysis~\cite{} evaluates
how prediction quality improves as low-confidence outputs are removed.
Confidence calibration~\cite{},
learned failure predictors~\cite{},
and out-of-distribution detectors~\cite{}
have been applied across vision tasks.
In pose estimation, failure detection has been explored
through confidence thresholding and heuristic geometric checks.
Yet these approaches typically rely on output magnitude
or require ground-truth failure labels for supervision,
rather than exploiting the geometric structure of the underlying inverse problem.
Our work derives trust signals
from the local conditioning of the 3D explanation itself,
requiring no additional annotations.

\paragraph{Inverse Problems and Conditioning.}
%
In classical inverse-problem theory~\cite{},
solution quality is characterized not only by residual error
but also by the conditioning of the forward operator.
Ill-conditioned problems are known to amplify observation noise,
yielding unstable solutions even when residuals are small.
This perspective has informed degeneracy analysis
in structure from motion~\cite{},
observability studies in visual-inertial odometry~\cite{},
and conditioning diagnostics for neural radiance fields~\cite{}.
For human pose, the depth ambiguity inherent in monocular 3D lifting
has long been recognized~\cite{},
but it is typically addressed by adding priors or multi-view constraints
rather than being quantified as a per-joint trust signal.
Our identifiability criterion formalizes this intuition:
we measure the sensitivity of the IK solution to input perturbations
and interpret ill-conditioning as geometric unreliability,
providing---to our knowledge---the first explicit use of conditioning analysis
as a trustworthiness measure for human pose estimation.
