\begin{abstract}
Human pose estimators often produce visually plausible predictions even when the underlying observations are unreliable. 
We argue that trustworthiness should be grounded not in heuristic confidence scores, but in the geometry of the underlying inverse problem.

We reinterpret 3D pose recovery as an inverse kinematics problem and assess trust at two complementary levels. 
First, we measure \emph{solvability}, evaluating whether a physically plausible 3D explanation exists for the observed 2D pose. 
Second, we introduce an \emph{identifiability} analysis that quantifies the conditioning of the inverse problem, measuring the sensitivity of the optimal 3D solution to small perturbations of the input joints.

While solvability tests whether a pose can be explained, identifiability tests whether it can be stably explained. 
We show that unreliable joints are characterized by ill-conditioned explanations even when reconstruction error is small. 
Experiments demonstrate that our approach significantly improves failure detection and risk-aware pose selection under occlusion and distribution shift, while maintaining competitive accuracy.
\end{abstract}
