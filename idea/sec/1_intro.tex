\section{Introduction}
\label{sec:intro}

Human pose estimation has reached impressive performance on standard benchmarks,
yet high accuracy does not imply reliability.
Modern estimators frequently produce confident predictions
even when visual evidence is weak, ambiguous, or absent.
Under occlusion, truncation, or distribution shift,
models may hallucinate geometrically plausible joints
that silently propagate errors to downstream systems
such as action recognition, autonomous driving, or clinical analysis.

The central challenge is therefore not only accuracy,
but \emph{trustworthiness}:
when should a predicted pose be believed?

Existing approaches attempt to answer this question through uncertainty estimation.
Heatmap peak responses~\cite{}, Monte Carlo dropout~\cite{}, and deep ensembles~\cite{}
measure dispersion within the model's representational space.
However, high confidence does not guarantee geometric validity.
A sharply peaked heatmap may correspond to an anatomically inconsistent configuration,
and multiple models may agree on a hallucinated joint.
Such methods quantify how certain the model is,
but not whether the prediction admits a physically coherent explanation.

A parallel line of work incorporates 3D reasoning and kinematic constraints
to improve pose accuracy,
including reprojection losses, inverse kinematics solvers,
and structured human body models.
These methods enforce geometric plausibility during training or inference,
yet they are primarily designed to refine predictions,
not to evaluate whether a prediction itself should be trusted.

In this work,  we connect these two perspectives. We ground pose trustworthiness in the geometry of the underlying inverse problem.
Every 2D pose implicitly asserts the existence of a 3D body configuration
that reproduces the observed joints under projection.
Trust should therefore be evaluated through the properties of this implied 3D explanation.

We cast 3D pose recovery as a regularized inverse kinematics (IK) problem
and derive two complementary criteria from its solution.

\begin{enumerate}[leftmargin=*,itemsep=2pt,topsep=2pt]
\item \textbf{Solvability.}
A predicted 2D pose is unreliable if no physically plausible 3D configuration can reproduce it.
We measure whether a low-energy IK solution exists,
and interpret large reprojection residuals as geometric inconsistency.

\item \textbf{Identifiability.}
Even when a low-energy solution exists,
the inverse problem may be ill-conditioned:
distinct 3D configurations can project to nearly identical 2D poses.
Using the implicit function theorem,
we show that the sensitivity of the optimal 3D solution
to infinitesimal perturbations of the input joints
is governed by the inverse Hessian of the IK objective.
High sensitivity indicates poor conditioning,
revealing joints that are weakly constrained by the observation,
even if reconstruction error is small.
\end{enumerate}

Solvability and identifiability capture fundamentally different failure modes.
A joint may be solvable yet weakly identified (e.g., depth-ambiguous limbs),
or identifiable yet unsolvable (e.g., joints outside the kinematic workspace).
Requiring both conditions yields a conservative and geometrically grounded trust measure.

Our framework requires no additional supervision.
Both trust signals are derived from the model's own predictions
and the kinematic structure of the human body.
A lightweight verifier head learns to regress joint-level trust scores
from these self-generated signals,
incurring negligible inference overhead.

Our contributions are as follows:
\begin{itemize}[leftmargin=*,itemsep=2pt,topsep=2pt]
\item We reinterpret pose trustworthiness as a property of the inverse problem underlying 3D explanation,
      distinguishing solvability (existence) and identifiability (conditioning).
\item We derive a conditioning-based characterization of joint reliability
      through the inverse Hessian of the IK objective,
      together with an efficient finite-difference approximation.
\item We demonstrate on COCO, OCHuman, and CrowdPose
      that our approach substantially improves failure detection
      and risk-aware pose selection under occlusion and distribution shift,
      while preserving competitive accuracy.
\end{itemize}
